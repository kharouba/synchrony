\documentclass[12pt,a4paper]{article}
\usepackage[top=1.00in, bottom=1.0in, left=1in, right=1in]{geometry}
\usepackage{amsmath}

\begin{document}

\noindent For the model \verb|twolevelrandomslope2.stan|, which has one hierachical level for species, represented here by $j$, each observation is $i$, and partial pooling on the slopes ($\beta$) only:

% no commas

\begin{equation}
y_{i}=\alpha_{j[i]}+\beta_{j[i]}X_{i}+\epsilon_{i}
\end{equation}

\begin{equation}
\beta_{j} \sim N(\mu_{\beta}, \sigma_{\beta})
\end{equation}

Which is the same as this:

\begin{equation}
y_{i}=\alpha_{sp[i]}+\beta_{sp[i]}X_{i}+\epsilon_{i}
\end{equation}

\begin{equation}
\beta_{sp} \sim N(\mu_{\beta}, \sigma_{\beta})
\end{equation}

Or this:

\begin{equation}
y_{i}=a_{j[i]}+b_{j[i]}X_{i}+\epsilon_{i}
\end{equation}

\begin{equation}
b_{j} \sim N(\mu_{b}, \sigma_{b})
\end{equation}

Or this:

\begin{equation}
y_{i}=\alpha_{j[i]}+\beta_{j[i]}(Year_{i}-1981)+\epsilon_{i}
\end{equation}

\begin{equation}
\beta_{j} \sim N(\mu_{\beta}, \sigma_{\beta})
\end{equation}

Or this:

\begin{equation}
y_{i}=\beta_{0j[i]}+\beta_{1j[i]}(Year_{i}-1981)+\epsilon_{i}
\end{equation}

\begin{equation}
\beta_{1j} \sim N(\mu_{\beta}, \sigma_{\beta})
\end{equation}

\vspace{3ex}
My personal favorite of all possible combinations:

\begin{equation}
y_{i}=\alpha_{sp[i]}+\beta_{sp[i]}(Year_{i}-1981)+\epsilon_{i}
\end{equation}

\begin{equation}
\beta_{sp} \sim N(\mu_{\beta}, \sigma_{\beta})
\end{equation}

\vspace{2ex}

\noindent Note that I did not write out the priors on $\mu_{\beta}$ and $\sigma_{\beta}$ and the $\alpha_{j}$, which are not specified and thus are uniform. I think you could just say in the text that we used the default priors. Also note that I think you could leave the $_{\beta}$ off $\mu$ and $\sigma$ if you really wanted, but perhaps nice for clarity?

\newpage

\noindent For the model \verb|threelevelrandomslope3.stan|, which has a hierachical level for species, nested within (another hierarchical level) study, and partial pooling on the slopes ($\beta$) and intercepts ($\alpha$) by species, but only pooling on study for the slopes; also, note that all species share a common $\sigma$ (for intercepts, $\sigma_{\alpha}$, and slopes, $\sigma_{\beta_{1}}$):

\begin{equation}
y_{i}=\alpha_{study[sp[i]]}+\beta_{study[sp[i]]}X_{i}+\epsilon_{i}
\end{equation}

\begin{equation}
\alpha_{study[sp]} \sim N(\alpha_{study}, \sigma_{\alpha})
\end{equation}


\begin{equation}
\beta_{study[sp]} \sim N(\beta_{study}, \sigma_{\beta_{1}})
\end{equation}


\begin{equation}
\beta_{study} \sim N(\mu_{\beta}, \sigma_{\beta_{2}})
\end{equation}

\begin{equation}
\sigma_{\alpha} \sim N(0,5)
\end{equation}

\begin{equation}
\sigma_{\beta_{1}} \sim N(0,5)
\end{equation}


\begin{equation}
\sigma_{\beta_{2}}\sim N(0,5)
\end{equation}

\begin{equation}
\mu_{\beta}\sim \text{ ...no prior specified}
\end{equation}

\vspace{2ex}
\noindent Again, many variants of this possible (see above). For example, you could do species, indexed here by $j$, nested within study, indexed by $k$ and get:

\begin{equation}
y_{i}=\alpha_{k[j[i]]}+\beta_{k[j[i]]}X_{i}+\epsilon_{i}
\end{equation}

\begin{equation}
\alpha_{k[j]} \sim N(\alpha_{k}, \sigma_{\alpha})
\end{equation}


\begin{equation}
\beta_{k[j]} \sim N(\beta_{k}, \sigma_{\beta_{1}})...
\end{equation}

\begin{equation}
\beta_{k} \sim N(\mu_{\beta}, \sigma_{\beta_{2}})...
\end{equation}

\vspace{2ex}

\noindent See also: 12.5 in Gelman \& Hill, pages 262-265.

\end{document}

\begin{equation}
\end{equation}